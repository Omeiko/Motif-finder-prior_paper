\section{Introduction}
Recently, the high-throughput chromosome conformation capture (Hi-C) technique has become a powerful tool for studying the three-dimensional structure of chromosomes. Hi-C data is usually expressed as an $n \times n$ matrix. The resolution of Hi-C data is defined as the bin size of each cell of the matrix. Hi-C data at kilobase level are requisite for future genome 3D structure studies. Rao et al.(2014) generated Hi-C data with 1 kilobase resolution. However, millions of sequenced reads are required to archive this resolution with a huge amount of money and time consumption.\\
Zhang et al. presented a approach to enhance the resolution of Hi-C data called HiCPlus. HiPlus generated low-resolution data by down-sampling the number of sequenced reads and then a neural network was used to create the mapping between high-resolution contact map and low-resolution contact map. 

\cite{Mizushima2011Autophagy}