\documentclass[uplatex,dvipdfmx]{jsarticle}
\begin{document}
\title{3Dゲノム コンタクト マップ作成手法の研究}
\author{王 明古と丸山 修}
\maketitle
\section{背景}
\section{目的}
\section{手法}
\section{結果}
\subsection{クロスバリデーション}
\subsection{外部から予測されたクロマチンコンタクトに対する検証}
\subsection{5Cデータに対する検証}
%我们使用我们的手法生成超高分辨率的coantact map,但是由于Hi-C数据的sequencing depth比较低,不容易找到全基因维度的高精度Hi-C数据来验证我们的结果。所以我们在这里使用5C数据,5C一般用于研究中等基因区域的DNA结构,由于5C方法的sequencing depth较高,能够得到更精确的相互作用数据。使用5C数据辅助可以验证我们的方法是否能够提升contact map的分辨率。
%首先我们需要找到同样cell type的Hi-C和5C数据,这里我们参考HIFI中使用的数据。1. 2.
%首先,我们计算了这两种数据中Hi-C与5C之间的相关度(相关系数,HIFI中使用スピアマン),我们定义为$$\alpha_1$$

我々の手法を使用して超高解像度のcontact mapを作成していますが、Hi-Cデータのsequencing depthが低いため、私たちの結果を検証するために、ゲノムワイドで高精度のHi-Cデータを見つけることは容易ではありませんでした。 5C法は一般的に中程度の長さの遺伝子領域のDNA構造を調べるために用いられており、5C法の方がsequencing depthが高いため、より精度が高い相互作用データを得ることができます。 5Cデータを使用することで、我々の方法がコンタクトマップの解像度を向上させることができるかどうかを検証することができます。

同じセルタイプのHi-Cと5Cのデータを見つける必要がありますが、ここではHIFIで使用されている二種類のデータを参照しています。(1)a 4-Mb region around the Xist gene in mouse embryonic stem cells(2)GM12878

まず、オリジナルのHi-Cと対応する5Cの相関関係(相関係数、HIFIでスピアマン相関係数を使用)を算出して、$\alpha_1$と定義します。我々の手法を用いて解像度を向上させた後のHi-Cデータと対応する5Cの相関係数は$\alpha_2$と定義します。$\alpha_1$と$\alpha_2$を比較すると
$$\alpha_1 \textless \alpha_2$$
我々の方法がコンタクトマップの解像度を向上させることができると評価できます。
$$\alpha_1 \textgreater \alpha_2$$
我々の方法がコンタクトマップの解像度を向上させることができると評価できません。
\end{document}