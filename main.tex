\documentclass{article}
% \usepackage[style=alphabetic]{biblatex}
\begin{document}
\title{Enhancing Hi-C contact map resolution with neural network}
\author{}
\maketitle

\section{Introduction}
Recently, the high-throughput chromoseome comformation capture(Hi-C) technique has become a powerful tool for studying the three-dimensional structure of chromosomes. Hi-C data is usually expressed as a $n \times n$ matrix. The resolution of Hi-C data is defined as the bin size of each cell of the matrix. Hi-C data at kilobase level are requisite for future genome 3D structure studies. Rao et al.(2014) generated Hi-C data with 1 kilobase resolution. However, millions of sequenced reads are required to archive this resolution with a huge amount of money and time consumption.\\
Zhang et al. presented a approach to enhance the resolution of Hi-C data called HiCPlus. Which generated low-resolution data by down-sampling the number of sequenced reads and then a neural network was used to create the mapping between high-resolution contact map and low-resolution contact map. 

\cite{Mizushima2011Autophagy}

\section{Methods}


Let $D$ be a set of paired ends reads of a Hi-C experiment. 

We make a low and high contact maps from $D$, denoted by $M_\ell$ and $M_h$. 

Let $S$ be the size of $M_\ell$ and $M_h$. 

% algorithm 環境にする.

\begin{verbatim}
% training part: 
for i = 1, 2, ..., $S-39$
    for j = 1, 2, ..., $S-39$
        extract sub-maps whose lefttop coordinate is (i,j) from $M_\ell$ and $M_h$. 

Let $C$ be a collection of the resulting sub-maps. 
Train a neural network using $C$. 

% test part: 
Use other chomosome. 

for i = 1, 2, ..., $S-39$
    for j = 1, 2, ..., $S-39$
        extract sub-maps whose lefttop coordinate is (i,j) from $M_\ell$ and $M_h$. 

\end{verbatim}






\subsubsection*{Step 1 Data preparation and processing}
Since this experiment is to validate the algorithm for mapping low-resolution data to high-resolution data, 
high-resolution data are required. 

In order to compare to some state-of-the-art approaches (HiCPlus and HiCNN), 
we use data sets (such as GM12878 from GSE63525) which are also used in other approaches. 
We start from generating a 10kb resolution contact map 
using Hi-C Pro. 
Then we perform down-sampling on high-resolution data. 
We use BAM files to generate low-resolution contact maps by changing the bin size bigger. 
We generate three contact maps with bin sizes are 20kb, 30kb and 40kb, respectively. 
We use chromosome 1-8 as training sets, and chromosome 17 as test set.

\subsubsection*{Step 2 Learning by Neural network}
We separate the low-resolution contact map into many $40 \times 40$ submatrices. 
Those submatrices are used as inputs.

\subsection{Layer Structure}
We consider the 




\bibliographystyle{acm}
\bibliography{ref} 

\end{document}