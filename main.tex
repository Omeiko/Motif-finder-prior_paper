\documentclass{article}
\usepackage{fancyvrb}
% \usepackage[style=alphabetic]{biblatex}
\begin{document}
\title{Enhancing Hi-C contact map resolution with neural network}
\author{Wang Minggu And Maruyama Osamu}
\maketitle

\section{introduction}


\section{Methods}


Let $D$ be a set of paired ends reads of a Hi-C experiment. 
We generate low and high contact maps from $D$, denoted by $M_\ell$ and $M_h$. 
Let $S_\ell$ and $S_h$ be the size of $M_\ell$ and $M_h$. 
Let $R$ be the ratio of $S_\ell$ to $S_h$, which can be represented by
$R = \frac{S_\ell}{S_h}$. And $P$ be the number of overlapping pixels between adjacent sub-maps.
Let $K$ be the size of low-resolution sub-maps.
% algorithm 環境にする.

\begin{Verbatim}[commandchars=\\\{\},codes={\catcode`$=3\catcode`_=8\catcode`^=7}]
% Divide matrices $M_\ell$ and $M_h$ \

To $M_\ell:$

for $i$ = 1, 1+$K-P$, 1+2$\times(K-P)$, ...
    for $j$ = 1, 1+$K-P$, 1+2$\times(K-P)$, ...
        IF $i+K$ > $M_\ell$ || $j+K$ > $M_\ell$, BREAK 
        ELSE extract K $\times$ K sub-maps whose left-top coordinate
         is $(i,j)$ from $M_\ell$.

Do the same process to $M_h:$

for $i$ = 1, 1+$(K-P) \times R$, 1+2$\times(K-P) \times R$, ...
    for $j$ = 1, 1+$(K-P) \times R$, 1+2$\times(K-P) \times R$, ...
        IF $i+K \times R$ > $M_h$ || $j+K \times R$ > $M_h$, BREAK 
        ELSE extract $(K \times R) \times (K \times R)$ sub-maps whose left-top 
        coordinate is $(i,j)$ from $M_h$.
\end{Verbatim}

% training part: 


\noindent Let $C_\ell$ and $C_h$ be collections of the resulting low-resolution and high-resolution
sub-maps. Let $\hat{C_h}$ be the collection of the high-resolution sub-matrices generated by neural network. We train a neural network using $C_\ell$ and $C_h$. The mean square error(MSE) is used as 
loss function in the training process. 

\begin{center}
    $MSE[C_\ell, C_h] = \frac{1}{K \times K} \sum_{i=1}^{K \times R} \sum_{j=1}^{K \times R} (\hat{C}_{h_{i,j}}-C_{h_{i,j}})^2$
\end{center}
Where $C_{\ell_{i,j}}$ and $C_{h_{i,j}}$ represent left-top coordinate $(i,j)$ in $C_\ell$ and $C_h$ respectively.
\noindent We can use $(f,n)$ to represent the parameters of each layer. Parameter $f$ means the size of the filter and $n$ means the number of filter. 

\noindent 1st Layer (Pattern extraction)
Base on every $K \times K$ sub-matrix, using $f \times f$( $13 \times 13$ in HiCPlus) filters
to extract patterns of each sub-contact-map. Which can represented by following formula:
\begin{center}
$F_1(X) = ReLU(w_1 * X + b_1)$
\end{center}
where $*$ represent the convolution process. $w_1$ represents $n \times f \times f$ filters. 
$b_1$ is the bias.
 

\noindent 2nd Layer (Low-res mapping to high-res)

\noindent 3rd Layer (Predicted contact maps generation)


% test part: 
Use other chromosome. 

Do the same dividing process like $M_\ell$ and $M_h$

Calculate the Pearson's correlation between the output and $M_h$.







\subsubsection*{Step 1 Data preparation and processing}
Since this experiment is to validate the algorithm for mapping low-resolution data to high-resolution data, 
high-resolution data are required. 

In order to compare to some state-of-the-art approaches (HiCPlus and HiCNN), 
we use data sets (such as GM12878 from GSE63525) which are also used in other approaches. 
We start from generating a 10kb resolution contact map 
using Hi-C Pro. 
Then we perform down-sampling on high-resolution data. 
We use BAM files to generate low-resolution contact maps by changing the bin size bigger. 
We generate three contact maps with bin sizes are 20kb, 30kb and 40kb, respectively. 
We use chromosome 1-8 as training sets, and chromosome 17 as test set.

\subsubsection*{Step 2 Learning by Neural network}
We separate the low-resolution contact map into many $K \times K$ sub-matrices. 
Those sub-matrices are used as inputs.

\subsection{Layer Structure}
We consider the 




\bibliographystyle{acm}
\bibliography{ref} 

\end{document}